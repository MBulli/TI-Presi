% !TeX encoding = UTF-8
\section{Unmöglichkeitsergebnisse}

\begin{frame}{Unmöglichkeitsergebnisse}
	 Wann stoßen Apprimationsschemata an ihre Grenzen?   
\end{frame}

\begin{frame}{Unmöglichkeitsergebnisse}
	Definition nach Wanka (S. 73, Definition 4.9) \newline
	
	Ein NP-vollständiges Entscheidungsproblem $L$ heißt stark NP-vollständig,  wenn es ein Polynom $q$ gibt, so dass $L_q = \{x\vert x \in L \text{ und } \maxnr(x) \leq q(\vert x \vert)\}$ NP-vollständig  ist. Gibt es kein solches Polynom, heißt $L$ schwach  NP-vollständig.
\end{frame}

\begin{frame}{Unmöglichkeitsergebnisse}
	Bedeutung
	\begin{itemize}
		\item Ist ein Entscheidungsproblem $L$ NP-vollständig und $\maxnr(x)$ polynomiell in der Länge der Eingabe $\vert x \vert$ beschränkt, so ist es stark NP-vollständig.
		\item D.h. trotz der Einschränkung, dass $\maxnr(x)$ nicht exponentiell werden darf, ist das Problem NP-vollständig und nicht in polynomieller Zeit lösbar.
		\item Gibt es kein solches Polynom, so ist das Problem $L$ schwach NP-vollständig.
	\end{itemize}
\end{frame}

\begin{frame}{Unmöglichkeitsergebnisse}
	Anders ausgedrückt
	\begin{itemize}
		\item Gibt es für ein NP-vollständiges Entscheidungsproblem $L$ keinen pseudopolynomiellen exakten Algorithmus, so ist das Problem stark NP-vollständig
		\item Voraussetzung: P $\neq$ NP\newline
	\end{itemize}
	\pause

	Enge Beziehung zwischen starker NP-Vollständigkeit und FPAS
	\begin{itemize}
		\item Gibt es ein Polynom $q(x_1,x_2)$, so dass für alle Probleminstanzen $I$ gilt, dass $\OPT(I) \le q(\vert I \vert, \maxnr(I))$ ist, dann folgt aus der Existenz eines FPAS für das Problem, dass es dafür auch einen pseudopolynomiellen exakten Algorithmus gibt.
		\item Gibt es für eine Optimierungsvariante eines stark NP-vollständigen Problems ein FPAS, dann ist P $=$ NP.
	\end{itemize}
\end{frame}
\begin{frame}
	Schlussfolgerung
	\begin{itemize}
		\item Ist N $\neq$ NP, so kann es für viele Optimierungsprobleme (\textsc{Clique}, Graphenfärbungsprobleme, TSP) keine streng polynomiellen Approximationsschemata geben. 
	\end{itemize}
	
\end{frame}