% !TeX encoding = UTF-8
\section{Pseudopolynomielle Algorithmen}

\begin{frame}{Pseudopolynomielle Algorithmen}
Definition:

Sei $\Pi$ ein kombinatorisches Optimierungsproblem, so dass für alle Instanzen $I$ gilt, dass alle in $I$ vorkommenden Zahlen natürliche Zahlen sind. Sei maxnr($I$) die größte in $I$ vorkommende Zahl. Ein Algorithmus für $\Pi$ heißt pseudopolynomiell, falls es ein Polynom poly(.,.)  gibt, so 
dass für alle Instanzen $I$ seine Laufzeit poly(|$I$| ,maxnr($I$)) ist. 

\end{frame}
\begin{frame}
Bedeutung:

\begin{itemize}
\item
Die Laufzeit ist polynomiell in der Eingabelänge und der größten vorkommenden Zahl beschränkt

\item
Das Problem ist (unter der Annahme P $\neq$ PN) nur für Eingaben mit großen Zahlen schwer lösbar, sonst effizient

\end{itemize}

Verwendung:
\begin{itemize}
\item
Macht NP-Vollständige Probleme unter gewissen Einschränkungen effizient lösbar

\end{itemize}
\end{frame}

\begin{frame}
\textbf{Verständliches Beispiel:}
\textbf{Primzahltest}

\begin{itemize}
\item
Testet, ob eine gegebene Zahl eine Primzahl ist
\item
Klassisches Verfahren: Teile die Eingabe $n$ durch alle Primzahlen mit $2 \leq p \leq \sqrt{n}$
\end{itemize}

\textbf{Eingabe:} Zahl $n$ \newline
\textbf{Ausgabe:} $true$ falls n zusammengesetzt wurde, $false$ sonst.\newline

\begin{algorithmic}

\For{Primzahlen $p$ mit $2 \leq p \leq \sqrt{n}$}
\If{$n \mod{p} = 0$}
	\State return $true$
\Else
	\State return $false$
\EndIf
\EndFor
\end{algorithmic}
\end{frame}

\begin{frame}
Problem:

\begin{itemize}
\item
Benötigt n-2 Divisionen, die laufzeit ist also linear in n
\item
Aber: Komplexität wird in der Eingabelänge |$n$| berechnet
\item
Daher: Exponentielle Laufzeit, problematisch bei großen Eingaben
TODO: Erklären warum man Eingabelänge betrachtet (Mehr Bits usw!)
\end{itemize}
\end{frame}

\begin{frame}
Anwendung auf \textsc{Rucksack}
\newline

Algorithmus: \newline

\begin{algorithmic}
\State $\alpha:=0;$
\Repeat
	\State $\alpha:=\alpha+1;$
	\For{$j:=1$ \textbf{to} $n$}
		\State $F_j(\alpha):=\min\{F_{j-1}(\alpha-p_j)+\text{vol}(j),F_{j-1}(\alpha)\};$
	\EndFor
\Until{$B < F_n(\alpha)$}
\State gib $\alpha-1$ aus$;$
\end{algorithmic}

\end{frame}