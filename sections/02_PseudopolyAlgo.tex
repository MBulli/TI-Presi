% !TeX encoding = UTF-8
\section{Pseudopolynomielle Algorithmen}

\begin{frame}{Pseudopolynomielle Algorithmen}
Was bedeutet pseudopolynomiell?

\begin{itemize}
	\item Eigenschaft eines Algorithmus
	
	\item Behauptung: Laufzeit ist abhängig von der Länge der Eingabe
	\item \textbf{Beispiel Sortieralgorithmus}
	\begin{itemize}
		\item Eingabe: Liste mit Elemente die sortiert werden sollen
		\item Problemgröße: Anzahl der Elemente in der Liste $\widehat{=}$ Länge der Eingabe
		\item	Sortieralgorithmen sind also \textbf{nicht} pseudopolynomiell, da die Eingabe bereits die Anzahl der elemente ist, die eigentliche darstellung der Elemente ist irrelevant
	\end{itemize}
	
\end{itemize}
\end{frame}

\begin{frame}{Pseudopolynomielle Algorithmen}
\textbf{Weiteres Beispiel: Primzahltest}

\begin{itemize}
	\item Testet, ob eine gegebene Zahl eine Primzahl ist
	\item Klassisches Verfahren: Teile die Eingabe $n$ durch alle ganzen Zahlen $p < \sqrt{n}$
\end{itemize}

\begin{algorithm}[H]
\caption{Naiver Primzahltest}
    \begin{algorithmic}
        \Require{ Zahl $n$}
        \Ensure{$true$ falls $n$ durch $p$ teilbar, $false$ sonst.}
        \Function{IsPrim}{$n$}
            \For{Zahl $p$ mit $2 \leq p \leq \sqrt{n}$}
            \If{$n \mod{p} = 0$}
                  	\State return $true$
            \Else
                  	\State return $false$
            \EndIf
            \EndFor
        \EndFunction
    \end{algorithmic}
\end{algorithm}
\end{frame}

\begin{frame}{Pseudopolynomielle Algorithmen}
Problem:

\begin{itemize}
	\item Benötigt naiv $n-2$ Divisionen.
	\item Aber: Komplexität wird in der Eingabelänge |$n$| berechnet, hier die anzahl an benötigten Bits
	\item Die Komplexität des Algorithmus hängt also logarithmisch von n ab, daher hat der Algorithmus eine exponentielle Laufzeit $\mathcal{O}(2^{|n|}) = \mathcal{O}(2^{\log_2{(n)}})$
	\item Daher ist der naive Primzahltest Algorithmus pseudopolynomiell
\end{itemize}
\end{frame}

\begin{frame}{Pseudopolynomielle Algorithmen}
Anwendung auf \textsc{Rucksack}
\newline

\begin{algorithm}[H]
    \caption{Exakter \rucksack/ Algorithmus}
        \begin{algorithmic}
            \State $\alpha:=0;$
            \Repeat
            \State $\alpha:=\alpha+1;$
            \For{$j:=1$ \textbf{to} $n$}
            \State $F_j(\alpha):=\min\{F_{j-1}(\alpha-p_j)+\text{vol}(j),F_{j-1}(\alpha)\};$
            \EndFor
            \Until{$B < F_n(\alpha)$}
            \State gib $\alpha-1$ aus$;$
        \end{algorithmic}
\end{algorithm}

\end{frame}

\begin{frame}{Pseudopolynomielle Algorithmen}
Bereits bekannt: Komplexität $\Rightarrow \mathcal{O}( n \cdot \OPT(I)) = \mathcal{O}( n^2 \cdot P_{\max})$ 
\begin{itemize}
	\item
	Aber: Worst-Case $: P_{max} = \OPT(I) = \Theta(2^{|I|})$
	\item
	Daher: 
	$\mathcal{O}(n\cdot 2^{|I|})$
	\item
	Die Laufzeit wird also exponentiell in der Eingabelänge
	\item
	Wenn die Eingabelänge polynomiell ist, dann läuft auch der Algorithmus polynomiell. Ist die Eingabelänge exponentiell groß, dann ist auch der Algorithmus exponentiell.
	\item $\Rightarrow$ Pseudopolynomieller Algorithmus
\end{itemize}

\end{frame}

\begin{frame}{Pseudopolynomielle Algorithmen}
Definition:

Sei $\Pi$ ein kombinatorisches Optimierungsproblem, so dass für alle Instanzen $I$ gilt, dass alle in $I$ vorkommenden Zahlen natürliche Zahlen sind. Sei maxnr($I$) die größte in $I$ vorkommende Zahl. Ein Algorithmus für $\Pi$ heißt pseudopolynomiell, falls es ein Polynom poly(.,.)  gibt, so 
dass für alle Instanzen $I$ seine Laufzeit poly(|$I$| ,maxnr($I$)) ist. 

\end{frame}
\begin{frame}{Pseudopolynomielle Algorithmen}
Bedeutung:

\begin{itemize}
	\item Die Laufzeit ist polynomiell in der Eingabelänge und der größten vorkommenden Zahl beschränkt
	\item Das Problem ist (unter der Annahme P $\neq$ PN) nur für Eingaben mit großen Zahlen schwer lösbar, sonst effizient
\end{itemize}

Verwendung:
\begin{itemize}
	\item Macht NP-Vollständige Probleme unter gewissen Einschränkungen effizient lösbar
\end{itemize}
\end{frame}
