% !TeX encoding = UTF-8
\section{Pseudopolynomielle Algorithmen}

\begin{frame}{Pseudopolynomielle Algorithmen}
Was bedeutet pseudopolynomiell?

\begin{itemize}
   	\item Eigenschaft eines Algorithmus
	\item Die Komplexität wird durch ein Polynom nach oben begrenzt, dass von zwei Variablen abhängt
    \begin{itemize}
        \item Die Länge der Eingabe $\abs{I}$ und
        \item die Länge der größten vorkommenden Zahl $\maxnr(I)$
    \end{itemize}
    \item Die Besonderheit ist hier, dass die Kodierung der Zahlen eine Rolle spielt
    \item Nur nummerische Algorithmen können pseudopolynomiell sein
    \item Jeder polynomieller Algorithmus ist auch pseudopolynomiell, aber nicht umgekehrt
\end{itemize}
\end{frame}

\begin{frame}{Pseudopolynomielle Algorithmen - Gegenbeispiel}
    \begin{itemize}
        \item \textbf{Fakt:} \textsc{Bubblesort} hat eine Laufzeit von $\mathcal{O}(n^2)$
        \item \textbf{Behauptung:} \textsc{Bubblesort} ist echt-polynomiell
        \item \textbf{Beweis:}
        \begin{itemize}
            \item Gegeben ist eine Liste mit $n$ Integers
            \item Die Eingabelänge in Bits entspricht somit $x=32n$
            \item Für die Komplexität ergibt sich
            \begin{itemize}
                \item $\mathcal{O}(x^2)=\mathcal{O}((32n)^2)=\mathcal{O}(1024n^2)=\mathcal{O}(n^2)$      
            \end{itemize}
            \item[] $\Rightarrow$ \textsc{Bubblesort} ist echt-polynomiell da die Laufzeit nur von der Anzahl der Elemente abhängt
        \end{itemize}
    \end{itemize}
\end{frame}

\begin{frame}{Pseudopolynomielle Algorithmen}
    Wie sähe ein \textsc{Bubblesort} aus der pseudopolynomiell ist?
    
\end{frame}

\begin{frame}{Pseudopolynomielle Algorithmen - \textsc{BubblesortPseudo}}
	
	\textsc{BubblesortPseudo} wird ein Integer $a$ übergeben. Intern wird ein Array mit der Länge $a$ erzeugt. Anschließend wird das Array zum Sortieren an \textsc{Bubblesort} übergeben.
	
	\begin{itemize}
		
		\item \textbf{Fakt:} \textsc{BubblesortPseudo} hat auch eine Laufzeit von $\mathcal{O}(a^2)$
		\item \textbf{Behauptung:} \textsc{BubblesortPseudo} ist pseudopolynomiell
		\pause
		\item \textbf{Beweis:} 
		\begin{itemize}
			\item Gegeben ist ein Integer $a$ 
			\item Die Eingabelänge in Bits entspricht somit $x = \lceil \log_2(a) \rceil$
			\item Die größte durch $x$ Bits darstellbare Zahl (Obere Grenze) ist $2^x$
			\item Für die Komplexität ergibt sich
			\begin{itemize}
				\item $\displaystyle \landau{a^2} = \landau{(2^x)^2}$
			\end{itemize}
			\item[] $\Rightarrow$ \textsc{BubblesortPseudo} hat eine Laufzeit die polynomiell von der übergebenen Zahl $a$ abhängt, jedoch pseudopolynomiell zum Verhältnis der Eingabelänge $x$ ist
		\end{itemize}
	\end{itemize}
\end{frame}

\begin{frame}{Pseudopolynomielle Algorithmen - \textsc{Primzahltest}}
	\begin{itemize}
		\item Testet, ob eine gegebene Zahl eine Primzahl ist
		\item Naives Verfahren: Teile die Eingabe $n$ durch alle ganzen Zahlen $\{2,3,...,n\}$
	\end{itemize}
	
	\begin{algorithm}[H]
		\caption{Naiver \textsc{Primzahltest}}
		\begin{algorithmic}
			\Require{Natürliche Zahl $n$}
			\Ensure{$true$ falls $n$ eine Primzahl ist, sonst $false$}
			\Function{IsPrim}{$n$}
			\For{$p=2$ \textbf{to} $n$}
			\If{$n \Mod{p} = 0$} 
			\State return $false$
			\EndIf
			\EndFor
			\State return $true$
			\EndFunction
		\end{algorithmic}
	\end{algorithm}
\end{frame}

\begin{frame}{Pseudopolynomielle Algorithmen}
	Problem:
	
	\begin{itemize}
		\item Benötigt naiv $n-2$ Divisionen, somit ist die Laufzeit $\landau{n}$ 
		\item Aber: Komplexität wird in der Eingabelänge $|n|$ berechnet, hier die Anzahl an benötigten Bits $x = \log_2(n)$ 
		\item Für die Komplexität ergibt sich
		\begin{itemize}
			\item $\displaystyle \landau{n^2} = \landau{(2^x)^2}$
		\end{itemize}
		\item $\Rightarrow$ \textsc{Primzahltest} Algorithmus ist pseudopolynomiell
	\end{itemize}
\end{frame}

\begin{frame}{Pseudopolynomielle Algorithmen}
Anwendung auf \textsc{Rucksack}
\newline

\begin{algorithm}[H]
    \caption{Exakter \rucksack/ Algorithmus}
        \begin{algorithmic}
            \State $\alpha:=0;$
            \Repeat
            \State $\alpha:=\alpha+1;$
            \For{$j:=1$ \textbf{to} $n$}
            \State $F_j(\alpha):=\min\{F_{j-1}(\alpha-p_j)+\text{vol}(j),F_{j-1}(\alpha)\};$
            \EndFor
            \Until{$B < F_n(\alpha)$}
            \State gib $\alpha-1$ aus$;$
        \end{algorithmic}
\end{algorithm}

\end{frame}

\begin{frame}{Pseudopolynomielle Algorithmen}
Bereits bekannt: Komplexität $\Rightarrow \mathcal{O}( n \cdot \OPT(I)) = \mathcal{O}( n^2 \cdot P_{\max})$ 
\begin{itemize}
	\item
	Aber: Worst-Case $: P_{max} = \OPT(I) = \Theta(2^{|I|})$
	\item
	Daher: 
	$\mathcal{O}(n\cdot 2^{|I|})$
	\item
	Die Laufzeit wird also exponentiell in der Eingabelänge
	\item
	Wenn die Eingabelänge polynomiell ist, dann läuft auch der Algorithmus polynomiell. Ist die Eingabelänge exponentiell groß, dann ist auch der Algorithmus exponentiell.
	\item $\Rightarrow$ Pseudopolynomieller Algorithmus
\end{itemize}

\end{frame}

\begin{frame}{Pseudopolynomielle Algorithmen - Definition}

Definition nach Wanka (S.70, Definition 4.5): \newline

Sei $\Pi$ ein kombinatorisches Optimierungsproblem, so dass für alle Instanzen $I$ gilt, dass alle in $I$ vorkommenden Zahlen natürliche Zahlen sind. Sei maxnr($I$) die größte in $I$ vorkommende Zahl. Ein Algorithmus für $\Pi$ heißt pseudopolynomiell, falls es ein Polynom poly(.,.)  gibt, so 
dass für alle Instanzen $I$ seine Laufzeit poly(|$I$| ,maxnr($I$)) ist. 

\end{frame}
\begin{frame}{Pseudopolynomielle Algorithmen}
Bedeutung:

\begin{itemize}
	\item Die Laufzeit ist polynomiell in der Eingabelänge und der größten vorkommenden Zahl beschränkt
	\item Das Problem ist (unter der Annahme P $\neq$ NP) nur für Eingaben mit großen Zahlen schwer lösbar, sonst effizient
\end{itemize}

Verwendung:
\begin{itemize}
	\item Macht NP-Vollständige Probleme unter gewissen Einschränkungen effizient lösbar
\end{itemize}
\end{frame}
