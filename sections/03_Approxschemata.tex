% !TeX encoding = UTF-8
\section{Approximationsschemata}
\begin{frame}{Approximationsschemata}


\begin{itemize}
\item Bisher: Konkrete Algorithmen, welche die Lösung eines Optimierungsproblem bis zu einer gültigen Schranke annähern
\item Für jede vorgegebene Schranke kann ein Approximationsalgorithmus gefunden werden, was unbefriedigend ist
\pause
\item Einführung einer Fehlertoleranz $\varepsilon$
\item $\varepsilon$ ist die maximale relative Abweichung vom Optimalwert
\item \textbf{Approximationsschema}: Familie aller Approximationsalgorithmen, die ein Problem für $\varepsilon > 0$ lösen
\end{itemize}
\end{frame}

\begin{frame}{Approximationsschemata}	
    Definition:
			
Sei $\Pi$ ein Optimierungsproblem. Sei $A$ ein Approximationsalgoritmhus für $\Pi$, der als Eingabe eine Probleminstanz $I$ von $\Pi$ \textit{und} ein $\varepsilon$ mit $0 < \varepsilon < 1$ bekommt.

\begin{enumerate}
\item
$A$ ist ein \textbf{\textit{polynomielles Approximationsschema} (PAS; engl.: \textit{polynomial approximation scheme})} für $\Pi$, wenn \textit{A} zu jeder Probleminstanz $I$ und für jedes $\varepsilon \in ] 0,1 [$ in Zeit $\mathcal O(poly(|I|))$ eine zulässige Lösung zu \textit{I} mit relativem Fehler $\varepsilon_A(\textit{I},\varepsilon) \le \varepsilon$ berechnet.

\item
$A$ ist ein \textbf{\textit{streng polynomielles Approximationsschema} (FPAS; engl.: \textit{fully PAS})}, wenn $A$ ein PAS mit Laufzeit $\mathcal O(poly(|I|, \frac{1}{\varepsilon}))$ ist.

\end{enumerate}		      
\end{frame}

\begin{frame}{Entwicklung eines FPAS - Algorithmus $\text{AR}_k$ Überlegung}	
	\begin{itemize}
		\item 
		Laufzeit des \textsc{DynRucksack} Algorithmus: $\mathcal O(n^2 \cdot P_{\max})$ 
		\item
		Laufzeit ist abhängig von $P_{\max}$
	\end{itemize}
	$\Rightarrow$ Verringere $P_{\max}$ um die Laufzeit zu verbessern
\end{frame}

\begin{frame}{Entwicklung eines FPAS - Algorithmus $\text{AR}_k$ Vorgehen}	
	\begin{itemize}
		\item 
		Man reduziere alle Preise der Waren in $I$ um $k$ d.h. ersetze $p_j$ durch $\displaystyle \lfloor \frac {p_j}{k} \rfloor $.
		\item
		Löse die neue Eingabemenge $I_{red}$ mittels \textsc{DynRucksack} und erhalte Lösungsmenge $R_k$.
		\item
		Die Lösungsmenge $R_k$ ist eine zulässige Lösung für $I$, da die Änderung der Preise keine Auswirkung auf die Rucksackfüllung hat.
		
	\end{itemize}
	$\Rightarrow$ Laufzeit: $\displaystyle \mathcal O(n^2 \cdot \frac{P_{\max}}{k})$
\end{frame}

\begin{frame}{Entwicklung eines FPAS - Algorithmus $\text{AR}_k$ Nachteil}
	Durch die Ersetzung der Preise $p_j$ um $\displaystyle \lfloor \frac {p_j}{k} \rfloor $ ensteht ein (relativer) Fehler.
	\begin{itemize}
		
		\item
		$\displaystyle \varepsilon_{AR_k}(I) = \frac {\text{absolute Güte}} {\text{optimale Lösung}} = \frac{\vert AR_k(I) - \text{OPT}(I) \vert}{\text{OPT}(I)}$ \\~\\
		\pause
		\item
		$AR_k(I) = [...] = \text{OPT}(I) - k \cdot \vert R^* \vert \ge \text{OPT}(I) - k \cdot n$
		\\~\\
	\end{itemize}	
	
	$\Rightarrow \vert AR_k(I) - \text{OPT}(I) \vert \le k \cdot n$\\~\\
	\pause
	\begin{itemize}
		\item
		$\displaystyle \varepsilon_{AR_k}(I) = \frac{\vert AR_k(I) - \text{OPT}(I) \vert}{\text{OPT}(I)} \le \frac{k \cdot  n}{\text{OPT(I)}} \le \frac{k \cdot n}{P_{\max}}$
	\end{itemize}
	
	$\Rightarrow$ Relativer Fehler: $\displaystyle \varepsilon_{AR_k}(I) \le \frac{k \cdot n}{P_{\max}}$\\~\\
	
	Da das Rucksackproblem ein Maximierungsproblem ist kann der relative Fehler nicht größer als 1 sein.
\end{frame}

\begin{frame}{Entwicklung eines FPAS - FPAS Wiederhohlung}
	Vorraussetzungen für einen FPAS:
	\begin{enumerate}
		\item für alle $\varepsilon \in ]0, 1[$ 
		\item mit $\varepsilon_A(I,\varepsilon) \le \varepsilon$
		\item in $\displaystyle \mathcal O(poly(\vert I \vert, \frac{1}{\varepsilon}))$
	\end{enumerate}
\end{frame}

\begin{frame}{Entwicklung eines FPAS - Algorithmus \textsc{FPASRucksack} Überlegung}
	\begin{itemize}
		\item
		Suchen eines Algorithmus dessen Laufzeit abhängig vom übergebenen relativen Fehler ist.
		\item
		Der $AR_k$ Algorithmus liefert bereits ein Ergebnis mit einem relativen Fehler abhängig eines Wertes $k$.
	\end{itemize}
\end{frame}

\begin{frame}{Entwicklung eines FPAS - Algorithmus \textsc{FPASRucksack} Bestimmung von $k$}
	Bestimme Übergabeparameter $k$ des $AR_k$ anhand des übergebenen Fehlers $\varepsilon$.
	\begin{itemize}
		\item
		Relativer Fehler des $AR_k$: $\displaystyle \varepsilon_{AR_k}(I) \le \frac{k \cdot n}{P_{\max}}$
		\pause
		\item[]
		$\displaystyle \varepsilon \cdot P_{\max} \le k \cdot n$
		\item[]
		$\displaystyle \varepsilon \cdot \frac{P_{\max}}{n} \le k$
		\item[]
		$\displaystyle k := \varepsilon \cdot \frac{P_{\max}}{n}$
	\end{itemize}

$\Rightarrow \varepsilon$ kann frei $\in]0,1[$ gewählt werden und da $\varepsilon_{AR_k}(I) \le \varepsilon_{FPAS}(I,e)$  ist die 2. Bedingung erfüllt.

\end{frame}
\begin{frame}{Entwicklung eines FPAS - Algorithmus \textsc{FPASRucksack} Laufzeit}
	\begin{itemize}
		\item
		Laufzeit des $AR_k$: $\displaystyle \mathcal O(n^2 \cdot \frac{P_{\max}}{k})$
		\pause
		\item[]
		$\displaystyle \mathcal O(n^2 \cdot \frac{P_{\max}}{\varepsilon \cdot \frac{P_{\max}}{n}})$
		\item[]
		$\displaystyle \mathcal O(n^2 \cdot \frac{n \cdot P_{\max}}{\varepsilon \cdot P_{\max}})$
		\item[]
		$\displaystyle \mathcal O(n^3 \cdot \frac{1}{\varepsilon})$
	\end{itemize}
$\Rightarrow$ Laufzeit des Algorithmus ist somit gemäß der 3. Bedingung für FPAS $\displaystyle \mathcal O(poly(\vert I \vert, \frac{1}{\varepsilon}))$
\end{frame}