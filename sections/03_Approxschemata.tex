% !TeX encoding = UTF-8
\section{Approximationsschemata}

\begin{frame}{Approximationsschemata}	
    Definition:
			
Sei $\Pi$ ein Optimierungsproblem. Sei $A$ ein Approximationsalgoritmhus für $\Pi$, der als Eingabe eine Probleminstanz $I$ von $\Pi$ \textit{und} ein $\epsilon$ mit $0 < \epsilon < 1$ bekommt.

\begin{enumerate}
\item
$A$ ist ein \textit{polynomielles Approximationsschema} (PAS; engl.: \textit{polynomial approximation scheme}) für $\Pi$, wenn \textit{A} zu jeder Probleminstanz $I$ und für jedes $\epsilon \in ] 0,1 [$ in Zeit $\mathcal O(poly(|I|)$ eine zulässige Lösung zu \textit{I} mit relativem Fehler $\epsilon_A(\textit{I},\epsilon) \leqslant \epsilon$ berechnet.

\item
$A$ ist ein \textit{streng polynomielles Approximationsschema} (FPAS; engl.: \textit{fully PAS}), wenn $A$ ein PAS mit Laufzeit $\mathcal O(poly|I|, \frac{1}{\epsilon}$) ist.

\end{enumerate}		      
\end{frame}