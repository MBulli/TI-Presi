% !TeX encoding = UTF-8
\section{Approximationsschemata}

\begin{frame}{Approximationsschemata}	
    Definition:
			
Sei $\Pi$ ein Optimierungsproblem. Sei $A$ ein Approximationsalgoritmhus für $\Pi$, der als Eingabe eine Probleminstanz $I$ von $\Pi$ \textit{und} ein $\epsilon$ mit $0 < \epsilon < 1$ bekommt.

\begin{enumerate}
\item
$A$ ist ein \textit{polynomielles Approximationsschema} (PAS; engl.: \textit{polynomial approximation scheme}) für $\Pi$, wenn \textit{A} zu jeder Probleminstanz $I$ und für jedes $\epsilon \in ] 0,1 [$ in Zeit $\mathcal O(poly(|I|))$ eine zulässige Lösung zu \textit{I} mit relativem Fehler $\epsilon_A(\textit{I},\epsilon) \leqslant \epsilon$ berechnet.

\item
$A$ ist ein \textit{streng polynomielles Approximationsschema} (FPAS; engl.: \textit{fully PAS}), wenn $A$ ein PAS mit Laufzeit $\mathcal O(poly|I|, \frac{1}{\epsilon}$) ist.

\end{enumerate}		      
\end{frame}

\section{Algorithmus $\text{AR}_k$}
\begin{frame}{Algorithmus $\text{AR}_k$ Überlegung}	
	\begin{itemize}
		\item 
		Laufzeit des \textsc{DynRucksack} Algorithmus: $\mathcal O(n^2 \cdot P_{\max})$ 
		\item
		Laufzeit ist abhängig von $P_{\max}$
		\item
		$\Rightarrow$ Verringere $P_{\max}$ um die Laufzeit zu verbessern
	\end{itemize}
\end{frame}
\begin{frame}{Algorithmus $\text{AR}_k$ Vorgehen}	
	\begin{itemize}
		\item 
		Man reduziere alle Preise der Waren in $I$ um $k$ d.h. ersetze $p_j$ durch $\displaystyle \lfloor \frac {p_j}{k} \rfloor $
		\item
		Löse die neue Eingabemenge $I_{red}$ mittels \textsc{DynRucksack} und erhalte Lösungsmenge $R_k$
		\item
		Die Lösungsmenge $R_k$ ist eine zulässige Lösung für $I$
		
	\end{itemize}
	$\Rightarrow$ Laufzeit: $\displaystyle \mathcal O(n^2 \cdot \frac{P_{\max}}{k})$
\end{frame}

\begin{frame}{Algorithmus $\text{AR}_k$ Nachteil}	
	\begin{itemize}
		\item 
		Durch die Ersetzung der Preise $p_j$ um $\displaystyle \lfloor \frac {p_j}{k} \rfloor $ ensteht ein (relativer) Fehler.
		
	\end{itemize}
	$\Rightarrow$ Relativer Fehler: $\varepsilon_{AR_k}(I) \le \frac{k \cdot n}{P_{\max}}$
\end{frame}