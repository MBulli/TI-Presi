% !TeX encoding = UTF-8
\section{Approximationsschemata}
\begin{frame}{Approximationsschemata}


\begin{itemize}
\item Bisher: Konkrete Algorithmen, die eine Lösung eines Optimierungsproblem bis zu einer gültigen Schranke annähern
\item Für jede vorgegebene Schranke kann ein Approximationsalgorithmus gefunden werden, was unbefriedigend ist
\item Einführung einer Fehlertoleranz $\epsilon$
\item $\epsilon$ ist die maximale relative Abweichung vom Optimalwert
\item \textbf{Approximationsschema}: Familie aller Approximationsalgorithmen, die ein Problem für $\epsilon > 0$ lösen
\end{itemize}
\end{frame}

\begin{frame}{Approximationsschemata}	
    Definition:
			
Sei $\Pi$ ein Optimierungsproblem. Sei $A$ ein Approximationsalgoritmhus für $\Pi$, der als Eingabe eine Probleminstanz $I$ von $\Pi$ \textit{und} ein $\epsilon$ mit $0 < \epsilon < 1$ bekommt.

\begin{enumerate}
\item
$A$ ist ein \textbf{\textit{polynomielles Approximationsschema} (PAS; engl.: \textit{polynomial approximation scheme})} für $\Pi$, wenn \textit{A} zu jeder Probleminstanz $I$ und für jedes $\epsilon \in ] 0,1 [$ in Zeit $\mathcal O(poly(|I|))$ eine zulässige Lösung zu \textit{I} mit relativem Fehler $\epsilon_A(\textit{I},\epsilon) \leqslant \epsilon$ berechnet.

\item
$A$ ist ein \textbf{\textit{streng polynomielles Approximationsschema} (FPAS; engl.: \textit{fully PAS})}, wenn $A$ ein PAS mit Laufzeit $\mathcal O(poly|I|, \frac{1}{\epsilon}$) ist.

\end{enumerate}		      
\end{frame}

\begin{frame}{Approximationsschemata}
Angabe eines FPAS für das Rucksackproblem
\begin{itemize}
\item Kann aus dem pseudopolynomiellen Algorithmus abgeleitet werden
\end{itemize}
\end{frame}
