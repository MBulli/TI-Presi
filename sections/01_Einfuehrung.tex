% !TeX encoding = UTF-8
\section{Einführung}

\begin{frame}{Einführung}
    Lorem ipsum dolor sit amet, consectetuer adipiscing elit. Maecenas porttitor congue massa. Fusce posuere, magna sed pulvinar ultricies, purus lectus malesuada libero, sit amet commodo magna eros quis urna.
    Nunc viverra imperdiet enim. Fusce est. Vivamus a tellus.
    Pellentesque habitant morbi tristique senectus et netus et malesuada fames ac turpis egestas. Proin pharetra nonummy pede. Mauris et orci.
    Aenean nec lorem. In porttitor. Donec laoreet nonummy augue.
    Suspendisse dui purus, scelerisque at, vulputate vitae, pretium mattis, nunc. Mauris eget neque at sem venenatis eleifend. Ut nonummy.      
\end{frame}
\begin{frame}{\textsc{Rucksack}-Problem}
    Grundidee: \qq{Qual der Wahl}
    
    \begin{itemize}
        \item Ein Dieb raubt einen Laden aus, jedoch hat er für die Beute nur einen Rucksack dabei
        \item Im Laden findet er $n$ Gegenstände
        \item Der $i$-te Gegenstand hat den Wert $p_i$ und das Gewicht $w$
        \item Sein Rucksack kann höchstens das Gewicht $B$ tragen
        \item $w_i$ und $B$ sind ganze Zahlen ($p_i$ können aus $\mathbb{R}^+$ sein)
    \end{itemize}
    \alert{Welche Gegenstände sollten für den maximalen Profit gewählt werden?} 
\end{frame}
\begin{frame}
    TODO Viz
\end{frame}
\begin{frame}{\textsc{Rucksack}-Problem formal}
    \begin{itemize}
        \item $\begin{aligned}[t] 
            \mathcal{D}=\{\langle W,\vol,p,B \rangle \mid & W=\{1,...,n\}, & \\
                                                                & \text{vol} \colon W \to \mathbb{N}, & \\
                                                                & p \colon W \to \mathbb{N}, & \\
                                                                & B \in \mathbb{N}, & \\
                                                                & \forall w \in W \colon \vol(w) \leq B \}
         \end{aligned}$
        
        \item $S( \langle W, \text{vol}, p, B \rangle )=\{ A \subseteq W \mid \sum_{w\in A}{ \vol(w)} \leq B \}$
        \item $f(A)=\sum_{w\in A}{p_w}$
        \item $\max$
    \end{itemize}
\end{frame}
\begin{frame}{0-1 \textsc{Rucksack}-Problem}
    maximiere $\displaystyle \sum_{i=1}^{n}{x_i p_i}$
       
    unter der Bedingung $\displaystyle \sum_{i=1}^{n}{x_i \vol(w_i) \leq B}$
       
    mit $x_i=1$ wenn Gegenstand $i$ im Rucksack enthalten ist, sonst $x_i=0$
\end{frame}
\begin{frame}{0-1 \textsc{Rucksack}-Problem}
    Maximaler Profit ohne Überschreitung des Rucksackvolumen.
    Idee: Profit diskret erhöhen und prüfen ob mehr gehen würde.
    Hierzu benötigen wir eine Funktion die für einen bestimmten Wert $\alpha$ das minimal benötige Volumen zurück gibt.
    So können wir prüfen ob unser Rucksackvolumen $B$ bei einem bestimmten Profit $\alpha$ überschritten ist.
    Dynamische Lösung: Schritt für Schritt an die optimale Lösung heran arbeiten.
\end{frame}
\begin{frame}{0-1 \textsc{Rucksack}-Problem}
    Für $j \in \{0,1,...,n\}$ und $\alpha \in \mathbb{Z}$ sei $F_j(\alpha)$ das kleinste benötigte Rucksackvolumen, mit dem
    man einen Wert von mindestens $\alpha$ erzielen kann, wenn man die ersten $j$ Waren einpacken darf. 
    
    \begin{equation*}
        F_j(\alpha) = \min\{\vol(R) \mid R \subseteq \{1,...,j\}, p(R) \geq \alpha  \}
    \end{equation*}
    
    Rekursion:
    \begin{equation*}
        F_j(\alpha) = \begin{cases}
        0 & \text{für } \alpha\leq 0, j \in \{0,...,n \} \\
        \infty & \text{für } \alpha\geq 1, j = 0 \\
        \min\{F_{j-1}(\alpha-p_j) + \vol(j), F_{j-1}(\alpha) \} & \text{sonst}
        \end{cases}       
    \end{equation*}
\end{frame}

\begin{frame}{Algorithmus \textsc{DynRucksack}}
    Gesucht ist insgesamt also das größte $\alpha$, sodass $F_n(\alpha)$ noch in den Rucksack der Kapazität $B$
    passt, d.h. $\OPT(I)=\max\{ \alpha \mid F_n(\alpha) \leq B \}$.
    
\begin{algorithm}[H]
    \caption{Exakter \rucksack/ Algorithmus}
        \begin{algorithmic}
            \State $\alpha:=0;$
            \Repeat
            \State $\alpha:=\alpha+1;$
            \For{$j:=1$ \textbf{to} $n$}
            \State $F_j(\alpha):=\min\{F_{j-1}(\alpha-p_j)+\text{vol}(j),F_{j-1}(\alpha)\};$
            \EndFor
            \Until{$B < F_n(\alpha)$}
            \State gib $\alpha-1$ aus$;$
        \end{algorithmic}
\end{algorithm}



\end{frame}
\begin{frame}{Komplexität von \textsc{DynRucksack}}
%    Innere Schleife: $n$-mal \\
%    Äußere Schleife: $\alpha$-mal \newline       
    $\mathcal{O}( n \cdot \alpha) = \mathcal{O}( n \cdot \OPT(I))$ \\~\\
    \pause    
    Es gilt $P_{\text{max}} \leq \text{OPT}(I) \leq n \cdot P_{\text{max}}$ 
    \quad mit $P_{\text{max}}=\max\{p_j \mid j \in \{1,...,n\}\}$ \\~\\
    \pause
    Warum?
    \begin{itemize}
        \item Untere Grenze: \\ Die minimale Rucksackfüllung beträgt $P_{\text{max}}$, da im schlimmsten Fall nur der wertvollste Gegenstand mitgenommen werden kann.
        \item Obere Grenze:  \\ Im Extremfall enthält die Rucksackfüllung alle $n$ Gegenstände, die alle den Preis $P_{\text{max}}$ haben. Somit ergibt sich der maximale Warenwert von $n \cdot P_{\text{max}}$.
    \end{itemize} ~\\
    
    $\Rightarrow \mathcal{O}( n \cdot n \cdot P_{\max}) = \mathcal{O}( n^2 \cdot P_{\max})$    
\end{frame}
