\documentclass[aspectratio=169,babel]{beamer}

\usetheme{boxes}

\usepackage[utf8]{inputenc}
\usepackage[T1]{fontenc}
\usepackage{bm}        % standard math notation (fonts)
\usepackage{amsfonts}
\usepackage{amssymb}
\usepackage{amsmath}  % standard math notation (vectors/sets/...)
\usepackage{commath}  % more math, \abs \norm
\usepackage[ngerman]{babel}
\usepackage[babel,german=quotes]{csquotes}
%\usepackage{listings}
\usepackage{graphicx}
%\usepackage{ulem}
\usepackage{hyperref}
\usepackage{graphicx} % eps graphics support
\usepackage{epsfig}
\usepackage{subfigure}
\usepackage{times}           % scalable fonts
\usepackage{docmute}
\usepackage{algorithm}
\usepackage{algpseudocode}	% for nice algorithm writing
\usepackage{soul}						% to strike through text \st

\newcommand{\qq}[1]{\glqq #1\grqq{}}


% Rename Require and Ensure labes to Input and Output
\renewcommand{\algorithmicrequire}{\textbf{Eingabe:}}
\renewcommand{\algorithmicensure}{\textbf{Ausgabe:}}

% mod without Space issue
\newcommand{\Mod}[1]{\ \text{mod}\ #1}

% Landau O
\newcommand{\landau}[1]{\mathcal{O}(#1)}


% Make Math serif
\usefonttheme{professionalfonts}
%\DeclareMathOperator{\vol}{vol}
\DeclareMathOperator{\OPT}{OPT}
\DeclareMathOperator{\maxnr}{maxnr}
\DeclareMathOperator{\Poly}{Poly}
\DeclareMathOperator{\Bitlength}{Bitlength}

\makeatletter
    \@ifdefinable{\rucksack}{\def\rucksack/{\textsc{Rucksack}}}
\makeatother

% Adds titles for each section using section name as title
\AtBeginSection[]{
  \begin{frame}
  \vfill
  \centering
  \begin{beamercolorbox}[sep=8pt,center,shadow=true,rounded=true]{title}
    \usebeamerfont{title}\insertsectionhead\par%
  \end{beamercolorbox}
  \vfill
  \end{frame}
}

\setbeamertemplate{footline}[page number]
\setbeamertemplate{section in toc}{\inserttocsectionnumber.~\inserttocsection}
%\setbeamertemplate{frametitle}[default][center]

\beamertemplatenavigationsymbolsempty

\title{Approximationsschemata}
\subtitle{Wege aus der NP-Vollständigkeit}
\author{Markus Bullmann, Thimo Eder, \\ Stefan Gerasch, Julius Hackel, Regis Kpokpoya}
\date{\today}

\begin{document}
    \maketitle
    \begin{frame}
        \frametitle{Gliederung}
        \tableofcontents
    \end{frame}
    
    % !TeX encoding = UTF-8
\section{Einführung}

\begin{frame}{Einführung}
	\begin{minipage}[t][0.25\textheight]{1\textwidth}
		\centering
		{\textit{\qq{I can't find an efficient algorithm, \\ but neither can all these famous people.}} \\ --- Garey und Johnson ---}
	\end{minipage}
	\pause
	\begin{minipage}[t][0.5\textheight]{1\textwidth}
	\begin{itemize}
		\item Annahme für alle folgenden Überlegungen: NP $\neq$ P
		\item Ein Problem ist NP-schwer, wenn es keinen Algorithmus gibt, der
			\begin{itemize}
				\item	deterministisch,
				\item exakt für alle Eingaben und 
				\item	effizient für alle Eingaben arbeitet.
			\end{itemize}
		\end{itemize}
	\end{minipage}
\end{frame}

\begin{frame}{Auswege aus der NP-Vollständigkeit}
\begin{itemize}
	\item In der Praxis sind NP-vollständige Problem jedoch trotzdem gut lösbar
	\item Auflockerung der obigen Bedingungen
	\item Effizient für \st{alle} gewisse Eingaben 
	\begin{itemize}
		\item[] $\Rightarrow$ Pseudopolynomielle Algorithmen
	\end{itemize}
	\item \st{Exakte} Approximierte Lösung
	\begin{itemize}
		\item[] $\Rightarrow$ Polynomielle Approximationsschemata
	\end{itemize}
\end{itemize}
\end{frame}

\begin{frame}{\textsc{Rucksack}-Problem}
    Grundidee: \qq{Qual der Wahl}
    
    \begin{itemize}
        \item Ein Dieb raubt einen Laden aus, jedoch hat er für die Beute nur einen Rucksack dabei
        \item Im Laden findet er $n$ Gegenstände
        \item Der $i$-te Gegenstand hat den Wert $p_i$ und das Gewicht $w_i$
        \item Sein Rucksack kann höchstens das Gewicht $B$ tragen
        \item $w_i$ und $B$ sind ganze Zahlen ($p_i$ können aus $\mathbb{R}^+$ sein)
    \end{itemize}
    \alert{Welche Gegenstände sollten für den maximalen Profit gewählt werden?} 
\end{frame}
\begin{frame}{Beispiel}
    \begin{figure}[ht]
    	\centering
    	\includegraphics[width=0.9\textwidth]{img/Rucksack.pdf}
    	\caption{Rucksackproblem}
    	\label{fig:rucksack}
    \end{figure}
\end{frame}
\begin{frame}{\textsc{Rucksack}-Problem formal}
    \begin{itemize}
        \item $\begin{aligned}[t] 
            \mathcal{D}=\{\langle W,\vol,p,B \rangle \mid & W=\{1,...,n\}, & \\
                                                                & \vol \colon W \to \mathbb{N}, & \\
                                                                & p \colon W \to \mathbb{N}, & \\
                                                                & B \in \mathbb{N}, & \\
                                                                & \forall w \in W \colon \vol(w) \leq B \}
         \end{aligned}$
        
        \item $S( \langle W, \vol, p, B \rangle )=\{ A \subseteq W \mid \sum_{w\in A}{ \vol(w)} \leq B \}$
        \item $f(A)=\sum_{w\in A}{p_w}$
        \item $\max$
    \end{itemize}
\end{frame}
\begin{frame}{0-1 \textsc{Rucksack}-Problem}
    maximiere $\displaystyle \sum_{i=1}^{n}{x_i p_i}$
       
    unter der Bedingung $\displaystyle \sum_{i=1}^{n}{x_i \vol(w_i) \leq B}$
       
    mit $x_i=1$ wenn Gegenstand $i$ im Rucksack enthalten ist, sonst $x_i=0$
\end{frame}
\begin{frame}{0-1 \textsc{Rucksack}-Problem}
    Maximaler Profit ohne Überschreitung des Rucksackvolumen.
    Idee: Profit diskret erhöhen und prüfen ob mehr gehen würde.
    Hierzu benötigen wir eine Funktion die für einen bestimmten Wert $\alpha$ das minimal benötige Volumen zurück gibt.
    So können wir prüfen ob unser Rucksackvolumen $B$ bei einem bestimmten Profit $\alpha$ überschritten ist.
    Dynamische Lösung: Schritt für Schritt an die optimale Lösung heran arbeiten.
\end{frame}
\begin{frame}{0-1 \textsc{Rucksack}-Problem}
    Für $j \in \{0,1,...,n\}$ und $\alpha \in \mathbb{Z}$ sei $F_j(\alpha)$ das kleinste benötigte Rucksackvolumen, mit dem
    man einen Wert von mindestens $\alpha$ erzielen kann, wenn man die ersten $j$ Waren einpacken darf. 
    
    \begin{equation*}
        F_j(\alpha) = \min\{\vol(R) \mid R \subseteq \{1,...,j\}, p(R) \geq \alpha  \}
    \end{equation*}
    
    Rekursion:
    \begin{equation*}
        F_j(\alpha) = \begin{cases}
        0 & \text{für } \alpha\leq 0, j \in \{0,...,n \} \\
        \infty & \text{für } \alpha\geq 1, j = 0 \\
        \min\{F_{j-1}(\alpha-p_j) + \vol(j), F_{j-1}(\alpha) \} & \text{sonst}
        \end{cases}       
    \end{equation*}
\end{frame}

\begin{frame}{Algorithmus \textsc{DynRucksack}}
    Gesucht ist insgesamt also das größte $\alpha$, sodass $F_n(\alpha)$ noch in den Rucksack der Kapazität $B$
    passt, d.h. $\OPT(I)=\max\{ \alpha \mid F_n(\alpha) \leq B \}$.
    
\begin{algorithm}[H]
    \caption{Exakter \rucksack/ Algorithmus}
        \begin{algorithmic}
            \State $\alpha:=0;$
            \Repeat
            \State $\alpha:=\alpha+1;$
            \For{$j:=1$ \textbf{to} $n$}
            \State $F_j(\alpha):=\min\{F_{j-1}(\alpha-p_j)+\text{vol}(j),F_{j-1}(\alpha)\};$
            \EndFor
            \Until{$B < F_n(\alpha)$}
            \State gib $\alpha-1$ aus$;$
        \end{algorithmic}
\end{algorithm}



\end{frame}
\begin{frame}{Komplexität von \textsc{DynRucksack}}
%    Innere Schleife: $n$-mal \\
%    Äußere Schleife: $\alpha$-mal \newline       
    $\mathcal{O}( n \cdot \alpha) = \mathcal{O}( n \cdot \OPT(I))$ \\~\\
    \pause    
    Es gilt $P_{\text{max}} \leq \text{OPT}(I) \leq n \cdot P_{\text{max}}$ 
    \quad mit $P_{\text{max}}=\max\{p_j \mid j \in \{1,...,n\}\}$ \\~\\
    \pause
    Warum?
    \begin{itemize}
        \item Untere Grenze: \\ Die minimale Rucksackfüllung beträgt $P_{\text{max}}$, da im schlimmsten Fall nur der wertvollste Gegenstand mitgenommen werden kann.
        \item Obere Grenze:  \\ Im Extremfall enthält die Rucksackfüllung alle $n$ Gegenstände, die alle den Preis $P_{\text{max}}$ haben. Somit ergibt sich der maximale Warenwert von $n \cdot P_{\text{max}}$.
    \end{itemize} ~\\
    
    $\Rightarrow \mathcal{O}( n \cdot n \cdot P_{\max}) = \mathcal{O}( n^2 \cdot P_{\max})$    
\end{frame}

    % !TeX encoding = UTF-8
\section{Pseudopolynomielle Algorithmen}

\begin{frame}{Pseudopolynomielle Algorithmen}
Was bedeutet pseudopolynomiell?

\begin{itemize}
   	\item Eigenschaft eines Algorithmus
	\item Die Komplexität wird durch ein Polynom nach oben begrenzt, dass von zwei Variablen abhängt
    \begin{itemize}
        \item Die Länge der Eingabe $\abs{I}$ und
        \item die Länge der größten vorkommenden Zahl $\maxnr(I)$
    \end{itemize}
    \item Die Besonderheit ist hier, dass die Kodierung der Zahlen eine Rolle spielt
    \item Nur nummerische Algorithmen können pseudopolynomiell sein
    \item Jeder polynomieller Algorithmus ist auch pseudopolynomiell, aber nicht umgekehrt
\end{itemize}
\end{frame}

\begin{frame}{Pseudopolynomielle Algorithmen - Gegenbeispiel}
    \begin{itemize}
        \item \textbf{Fakt:} \textsc{Bubblesort} hat eine Laufzeit von $\mathcal{O}(n^2)$
        \item \textbf{Behauptung:} \textsc{Bubblesort} ist echt-polynomiell
        \item \textbf{Beweis:}
        \begin{itemize}
            \item Gegeben ist eine Liste mit $n$ Integers
            \item Die Eingabelänge in Bits entspricht somit $x=32n$
            \item Für die Komplexität ergibt sich
            \begin{itemize}
                \item $\mathcal{O}(x^2)=\mathcal{O}((32n)^2)=\mathcal{O}(1024n^2)=\mathcal{O}(n^2)$      
            \end{itemize}
            \item[] $\Rightarrow$ \textsc{Bubblesort} ist echt-polynomiell da die Laufzeit nur von der Anzahl der Elemente abhängt
        \end{itemize}
    \end{itemize}
\end{frame}

\begin{frame}{Pseudopolynomielle Algorithmen}
    Wie sähe ein \textsc{Bubblesort} aus der pseudopolynomiell ist?
    
\end{frame}

\begin{frame}{Pseudopolynomielle Algorithmen - \textsc{BubblesortPseudo}}
	
	\textsc{BubblesortPseudo} wird ein Integer $a$ übergeben. Intern wird ein Array mit der Länge $a$ erzeugt. Anschließend wird das Array zum Sortieren an \textsc{Bubblesort} übergeben.
	
	\begin{itemize}
		
		\item \textbf{Fakt:} \textsc{BubblesortPseudo} hat auch eine Laufzeit von $\mathcal{O}(a^2)$
		\item \textbf{Behauptung:} \textsc{BubblesortPseudo} ist pseudopolynomiell
		\pause
		\item \textbf{Beweis:} 
		\begin{itemize}
			\item Gegeben ist ein Integer $a$ 
			\item Die Eingabelänge in Bits entspricht somit $x = \lceil \log_2(a) \rceil$
			\item Die größte durch $x$ Bits darstellbare Zahl (Obere Grenze) ist $2^x$
			\item Für die Komplexität ergibt sich
			\begin{itemize}
				\item $\displaystyle \landau{a^2} = \landau{(2^x)^2}$
			\end{itemize}
			\item[] $\Rightarrow$ \textsc{BubblesortPseudo} hat eine Laufzeit die polynomiell von der übergebenen Zahl $a$ abhängt, jedoch pseudopolynomiell zum Verhältnis der Eingabelänge $x$ ist
		\end{itemize}
	\end{itemize}
\end{frame}

\begin{frame}{Pseudopolynomielle Algorithmen - \textsc{Primzahltest}}
	\begin{itemize}
		\item Testet, ob eine gegebene Zahl eine Primzahl ist
		\item Naives Verfahren: Teile die Eingabe $n$ durch alle ganzen Zahlen $\{2,3,...,n\}$
	\end{itemize}
	
	\begin{algorithm}[H]
		\caption{Naiver \textsc{Primzahltest}}
		\begin{algorithmic}
			\Require{Natürliche Zahl $n$}
			\Ensure{$true$ falls $n$ eine Primzahl ist, sonst $false$}
			\Function{IsPrim}{$n$}
			\For{$p=2$ \textbf{to} $n$}
			\If{$n \Mod{p} = 0$} 
			\State return $false$
			\EndIf
			\EndFor
			\State return $true$
			\EndFunction
		\end{algorithmic}
	\end{algorithm}
\end{frame}

\begin{frame}{Pseudopolynomielle Algorithmen}
	Problem:
	
	\begin{itemize}
		\item Benötigt naiv $n-2$ Divisionen, somit ist die Laufzeit $\landau{n}$ 
		\item Aber: Komplexität wird in der Eingabelänge $|n|$ berechnet, hier die Anzahl an benötigten Bits $x = \log_2(n)$ 
		\item Für die Komplexität ergibt sich
		\begin{itemize}
			\item $\displaystyle \landau{n^2} = \landau{(2^x)^2}$
		\end{itemize}
		\item $\Rightarrow$ \textsc{Primzahltest} Algorithmus ist pseudopolynomiell
	\end{itemize}
\end{frame}

\begin{frame}{Pseudopolynomielle Algorithmen}
Anwendung auf \textsc{Rucksack}
\newline

\begin{algorithm}[H]
    \caption{Exakter \rucksack/ Algorithmus}
        \begin{algorithmic}
            \State $\alpha:=0;$
            \Repeat
            \State $\alpha:=\alpha+1;$
            \For{$j:=1$ \textbf{to} $n$}
            \State $F_j(\alpha):=\min\{F_{j-1}(\alpha-p_j)+\text{vol}(j),F_{j-1}(\alpha)\};$
            \EndFor
            \Until{$B < F_n(\alpha)$}
            \State gib $\alpha-1$ aus$;$
        \end{algorithmic}
\end{algorithm}

\end{frame}

\begin{frame}{Pseudopolynomielle Algorithmen}
Bereits bekannt: Komplexität $\Rightarrow \mathcal{O}( n \cdot \OPT(I)) = \mathcal{O}( n^2 \cdot P_{\max})$ 
\begin{itemize}
	\item
	Aber: Worst-Case $: P_{max} = \OPT(I) = \Theta(2^{|I|})$
	\item
	Daher: 
	$\mathcal{O}(n\cdot 2^{|I|})$
	\item
	Die Laufzeit wird also exponentiell in der Eingabelänge
	\item
	Wenn die Eingabelänge polynomiell ist, dann läuft auch der Algorithmus polynomiell. Ist die Eingabelänge exponentiell groß, dann ist auch der Algorithmus exponentiell.
	\item $\Rightarrow$ Pseudopolynomieller Algorithmus
\end{itemize}

\end{frame}

\begin{frame}{Pseudopolynomielle Algorithmen}
Definition:

Sei $\Pi$ ein kombinatorisches Optimierungsproblem, so dass für alle Instanzen $I$ gilt, dass alle in $I$ vorkommenden Zahlen natürliche Zahlen sind. Sei maxnr($I$) die größte in $I$ vorkommende Zahl. Ein Algorithmus für $\Pi$ heißt pseudopolynomiell, falls es ein Polynom poly(.,.)  gibt, so 
dass für alle Instanzen $I$ seine Laufzeit poly(|$I$| ,maxnr($I$)) ist. 

\end{frame}
\begin{frame}{Pseudopolynomielle Algorithmen}
Bedeutung:

\begin{itemize}
	\item Die Laufzeit ist polynomiell in der Eingabelänge und der größten vorkommenden Zahl beschränkt
	\item Das Problem ist (unter der Annahme P $\neq$ PN) nur für Eingaben mit großen Zahlen schwer lösbar, sonst effizient
\end{itemize}

Verwendung:
\begin{itemize}
	\item Macht NP-Vollständige Probleme unter gewissen Einschränkungen effizient lösbar
\end{itemize}
\end{frame}

    % !TeX encoding = UTF-8
\section{Approximationsschemata}
\begin{frame}{Approximationsschemata}


\begin{itemize}
\item Bisher: Konkrete Algorithmen, die eine Lösung eines Optimierungsproblem bis zu einer gültigen Schranke annähern
\item Für jede vorgegebene Schranke kann ein Approximationsalgorithmus gefunden werden, was unbefriedigend ist
\item Einführung einer Fehlertoleranz $\epsilon$
\item $\epsilon$ ist die maximale relative Abweichung vom Optimalwert
\item \textbf{Approximationsschema}: Familie aller Approximationsalgorithmen, die ein Problem für $\epsilon > 0$ lösen
\end{itemize}
\end{frame}

\begin{frame}{Approximationsschemata}	
    Definition:
			
Sei $\Pi$ ein Optimierungsproblem. Sei $A$ ein Approximationsalgoritmhus für $\Pi$, der als Eingabe eine Probleminstanz $I$ von $\Pi$ \textit{und} ein $\epsilon$ mit $0 < \epsilon < 1$ bekommt.

\begin{enumerate}
\item
$A$ ist ein \textbf{\textit{polynomielles Approximationsschema} (PAS; engl.: \textit{polynomial approximation scheme})} für $\Pi$, wenn \textit{A} zu jeder Probleminstanz $I$ und für jedes $\epsilon \in ] 0,1 [$ in Zeit $\mathcal O(poly(|I|))$ eine zulässige Lösung zu \textit{I} mit relativem Fehler $\epsilon_A(\textit{I},\epsilon) \leqslant \epsilon$ berechnet.

\item
$A$ ist ein \textbf{\textit{streng polynomielles Approximationsschema} (FPAS; engl.: \textit{fully PAS})}, wenn $A$ ein PAS mit Laufzeit $\mathcal O(poly|I|, \frac{1}{\epsilon}$) ist.

\end{enumerate}		      
\end{frame}

\section{Algorithmus $\text{AR}_k$}
\begin{frame}{Algorithmus $\text{AR}_k$ Überlegung}	
	\begin{itemize}
		\item 
		Laufzeit des \textsc{DynRucksack} Algorithmus: $\mathcal O(n^2 \cdot P_{\max})$ 
		\item
		Laufzeit ist abhängig von $P_{\max}$
		\item
		$\Rightarrow$ Verringere $P_{\max}$ um die Laufzeit zu verbessern
	\end{itemize}
\end{frame}
\begin{frame}{Algorithmus $\text{AR}_k$ Vorgehen}	
	\begin{itemize}
		\item 
		Man reduziere alle Preise der Waren in $I$ um $k$ d.h. ersetze $p_j$ durch $\displaystyle \lfloor \frac {p_j}{k} \rfloor $
		\item
		Löse die neue Eingabemenge $I_{red}$ mittels \textsc{DynRucksack} und erhalte Lösungsmenge $R_k$
		\item
		Die Lösungsmenge $R_k$ ist eine zulässige Lösung für $I$
		
	\end{itemize}
	$\Rightarrow$ Laufzeit: $\displaystyle \mathcal O(n^2 \cdot \frac{P_{\max}}{k})$
\end{frame}

\begin{frame}{Algorithmus $\text{AR}_k$ Nachteil}	
	\begin{itemize}
		\item 
		Durch die Ersetzung der Preise $p_j$ um $\displaystyle \lfloor \frac {p_j}{k} \rfloor $ ensteht ein (relativer) Fehler.
		
	\end{itemize}
	$\Rightarrow$ Relativer Fehler: $\varepsilon_{AR_k}(I) \le \frac{k \cdot n}{P_{\max}}$
\end{frame}

\begin{frame}{FPAS}
Angabe eines FPAS für das Rucksackproblem
	\begin{itemize}
		\item Kann aus dem pseudopolynomiellen Algorithmus abgeleitet werden
	\end{itemize}
\end{frame}

    % !TeX encoding = UTF-8
\section{Unmöglichkeitsergebnisse}

\begin{frame}{Unmöglichkeitsergebnisse}
	 Wann stoßen Apprimationsschemata an ihre Grenzen?   
\end{frame}

\begin{frame}{Unmöglichkeitsergebnisse}
	Definition nach Wanka (S. 73, Definition 4.9) \newline
	
	Ein NP-vollständiges Entscheidungsproblem $L$ heißt stark NP-vollständig,  wenn es ein Polynom $q$ gibt, so dass $L_q = \{x\vert x \in L \text{ und } \maxnr(x) \leq q(\vert x \vert)\}$ NP-vollständig  ist. Gibt es kein solches Polynom, heißt $L$ schwach  NP-vollständig.
\end{frame}

\begin{frame}{Unmöglichkeitsergebnisse}
	Bedeutung
	\begin{itemize}
		\item Ist ein Entscheidungsproblem $L$ NP-vollständig und $\maxnr(x)$ polynomiell in der Länge der Eingabe $\vert x \vert$ beschränkt, so ist es stark NP-vollständig.
		\item D.h. trotz der Einschränkung, dass $\maxnr(x)$ nicht exponentiell werden darf, ist das Problem NP-vollständig und nicht in polynomieller Zeit lösbar.
		\item Gibt es kein solches Polynom, so ist das Problem $L$ schwach NP-vollständig.
	\end{itemize}
\end{frame}

\begin{frame}{Unmöglichkeitsergebnisse}
	Anders ausgedrückt
	\begin{itemize}
		\item Gibt es für ein NP-vollständiges Entscheidungsproblem $L$ keinen pseudopolynomiellen exakten Algorithmus, so ist das Problem stark NP-vollständig
		\item Voraussetzung: P $\neq$ NP\newline
	\end{itemize}
	\pause

	Enge Beziehung zwischen starker NP-Vollständigkeit und FPAS
	\begin{itemize}
		\item Gibt es ein Polynom $q(x_1,x_2)$, so dass für alle Probleminstanzen $I$ gilt, dass $\OPT(I) \le q(\vert I \vert, \maxnr(I))$ ist, dann folgt aus der Existenz eines FPAS für das Problem, dass es dafür auch einen pseudopolynomiellen exakten Algorithmus gibt.
		\item Gibt es für eine Optimierungsvariante eines stark NP-vollständigen Problems ein FPAS, dann ist P $=$ NP.
	\end{itemize}
\end{frame}
\begin{frame}
	Schlussfolgerung
	\begin{itemize}
		\item Ist N $\neq$ NP, so kann es für viele Optimierungsprobleme (\textsc{Clique}, Graphenfärbungsprobleme, TSP) keine streng polynomiellen Approximationsschemata geben. 
	\end{itemize}
	
\end{frame}
    \input{sections/05_fazit.tex}

\end{document}